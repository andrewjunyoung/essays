\documentclass{article}

%% Begin package imports %%%%%%%%%%%%%%%%%%%%%%%%%%%%%%%%%%%%%%%%%%%%%%%%%%%%%%%

% Language and font encodings
\usepackage[english]{babel}
% Package for angle quotes etc.
\usepackage[T1]{fontenc}
% Set page size and margins
\usepackage[a4paper,top=3cm,bottom=2cm,left=3cm,right=3cm,marginparwidth=1.75cm]{geometry}
\usepackage[utf8]{inputenc}
\usepackage{hyperref}

%%%%%%%%%%%%%%%%%%%%%%%%%%%%%%%%%%%%%%%%%%%%%%%%%%%%%%%%% End package imports %%

\title{A fix for a timeless problem}
\author{Andrew J. Young}

\begin{document}
\maketitle

For all our history, we have been measuring time and modeling our world around
it. From rituals to seasons; scheduling reminders; and coordinating groups,
empirically measuring the passage of time, and our position in it, has been
vital.

In the modern world, time is embedded in the fabric of society. We travel on a
schedule, sleep on time, and make appointments at times. It's easy to overlook
it, but every day we work through hundreds of our own scheduled events.
International finance, computers, and cooperation have all become rooted in
accurate measurements of time, and understanding where we are in it, without
which the world would not be able to coordinate itself as it does.

Our calendar, and the way with which we deal with time, is imperfect. In the
gregorian calendar, the year starts at on an arbitrary day in winter, and has
leap days added to the second month. It has time zones that can stay the same;
change at any time; and be named the same as other zones. "Months" which once
aligned with the moon are not the same length, let alone the length of a lunar
cycle, and leap seconds need to be constantly added to avoid calendrical drift.

Despite this, through convention and ease of use, this calendar dominates, while
other calendars with less standardized features have drifted, been forked, or
rely on unreliable natural phenomena.

In the present, we need to deal with time zones, which don't have consistent
naming conventions, change, and not even on the same day. Managing the
complexity of time zones alone is unbelievably hard, as any programmer who has
tried to work with them will know.

In the future, we will need to find a way of keeping time on the moon and mars,
at least. Mars has 2 moons, not one, which have different phases. Some moons
have no phases. A year on mars is also longer than one on earth.

These problems aren't ideas that are 10,000 years in the future, they're ideas
that are happening now. we already have sent people to the moon, and have robots
on Mars. We need long term thinking to give a long term fix for these problems.

We do already have a fix for these problems: UNIX time. UNIX time is a
seconds-based counter from "epoch", defined to be 1970 January 1, at 00:00. The
arbitrariness of epoch time is not important (much like the arbitrariness of the
year 1), but is essential to be agreed upon. After this, we just count how many
seconds we are before or after epoch. This doesn't need to be seconds. Any scale
will do, providing its standardized, and ideally easy to use (that is,
decimalized).

In the scales of the universe which we will encounter in the next 10,000 years,
it seems that epoch time is a good absolute measure of the passage of time, and
can be used on earth and mars alike. But, it's not easy for humans to use, or to
understand.

Because of this, we have already been using 2 systems in parallel. While UNIX
time runs our computers, and governs absolute time as we know it, our calendars
are a human-friendly way of understanding our position in time, and where we're
going next. From these human-friendly calendars, like the gregorian calendar, we
can track years; days; and weeks: the fundamental building blocks of our
society.

At this point, we want to design an interface that will make sense on any
planet, so that a robot in mars can reasonably get a human on earth to
understand its situation. All this needs is years and days. Some planets have
many moons, and many moons have no phases. Planets do have differently lengthed
years and days, but these constants can be easily stored and referenced, and
they change very slowly.

This means that no matter what planet we're on, we can show time as:

YYYY a DDD d SSSSS s (eg. 2018 a 289 d 10,851 s)

Why no hours? We want to show a planet's time of day as where the sun's position
in the sky. If we define an hour as 3,600 s, then the idea of "noon" is
different on each planet. This is harder. If we define an hour as 1 / 24 of a
day, then hours are different lengths. This too is harder.

Under this system, all we need to know is

YYYY a DDD d (/ DDD d) SSSSS s (/SSSSS s)

That is, how many years have gone by on that planet since epoch; ¿how many days
have gone by since the start the current year (its shortest day)?; ¿Out of how
many days?; ¿how many seconds into that day are we? ¿out of how many seconds per
day?

This also allows us to quickly determine the day of the week: simply take the
day of the year d and calculate d (mod 7)!

If we're on a moon, we don't usually track seasons in the same way, because
these are determined both by the body we're orbiting and by the nearest star. In
this case, a year is most usefully represented to mean orbits around the local
planet. Hence, the year on the moon since epoch will be just over 13 times the
year since epoch on Earth, but this is what we want to see, because it helps us
better understand the conditions of the planet or rock which we're on.

And of course, our universal time is holding it all together.

\end{document}

%%%%%%%%%%%%%%%%%%%%%%%%%%%%%%%%%%%%%%%%%%%%%%%%%%%%%%% End document contents %%

